\documentclass[letter]{amsart}
\usepackage[utf8]{inputenc}
\usepackage{amsmath,amsfonts,amssymb,amsrefs,mathtools,amsthm}
\usepackage{enumerate}
\usepackage{graphicx}
\usepackage{multicol}
\usepackage[all]{xy}
\usepackage{tikz}
\usetikzlibrary{snakes,patterns}
\usetikzlibrary{decorations.pathreplacing,calligraphy}
\usepackage{caption}
\usepackage{cancel}
%\usepackage[T1]{fontenc}

\usepackage[bookmarksnumbered, colorlinks, plainpages]{hyperref}
\hypersetup{colorlinks=true,linkcolor=red, anchorcolor=green, citecolor=cyan, urlcolor=blue, filecolor=magenta, pdftoolbar=true}
\usepackage[capitalise]{cleveref}

\newtheorem*{definition}{Definition}
\newtheorem{theorem}{Theorem}
\newtheorem{lemma}[theorem]{Lemma}
\newtheorem{proposition}[theorem]{Proposition}
\newtheorem{remark}[theorem]{Remark}
\newtheorem{corollary}[theorem]{Corollary}
\newtheorem{example}[theorem]{Example}
\newtheorem{problem}[theorem]{Problem}
\newtheorem{Competition}[theorem]{Contest}


\newcommand{\AK}[1]{\marginpar{
	\begin{flushleft}{\color{red}\tiny #1}\end{flushleft}}}
\newcommand{\RJ}[1]{\marginpar{
\begin{flushleft}{\color{blue}\tiny #1}\end{flushleft}}}
\newcommand{\BM}[1]{\marginpar{
\begin{flushleft}{\color{green}\tiny #1}\end{flushleft}}}
%\newcommand{\kc}[1]{{\color[rgb]{1,0,0}#1}}
%\newcommand{\blue}[1]{{\color[rgb]{0,0,1}#1}}

\DeclareMathOperator{\atan2}{atan2}
\title[\ \ Draft of \today]{Group Theory, Summer 2025}

\author[Anon \ \ Draft of \today]{\textbf{}Anon}


%red, green, blue, cyan, magenta, yellow, black, gray, white, darkgray, lightgray, brown, lime, olive, orange, pink, purple, teal, violet.

\date{Draft of \today}
%00A05 Mathematics in general


\begin{document}
\maketitle
\tableofcontents

\section{\textbf{Introduction}}
Theoretical problems primarily from \cite{rotman1995groups}.\\
Computational problems primarily from \cite{moon2014homework}.See the url in the bibtex to find the actual uploads of the homework.
Contest problems are from various sources.\\
\section{\textbf{Groups and Homomorphism}}
\subsection{\text{Semigroups} $\oplus$ \text{Groups}}

\begin{problem} (1.23)
If $G$ is a group and $a_1, a_2, \dots, a_n \in G$, then
$$(a_1 a_2 \cdots a_n)^{-1} = a_n^{-1} a_{n-1}^{-1} \cdots a_1^{-1}.$$
Conclude that if $n \geq 0$, then
$$(a^{-1})^n =  (a^n)^{-1}.$$
\end{problem}
\begin{proof}
Let $P(n)$:$$(a_1 a_2 \cdots a_n)^{-1} = a_n^{-1} a_{n-1}^{-1} \cdots a_1^{-1},~~~~~ n\in \mathbb{N}$$\\

\AK{Strategy: Involves $n \in \mathbb{N}$ so induction.}

$P(1)$: $(a_1)^{-1}=a_1^{-1}$ is true.\\

Assume $P(k)$: $(a_1 \cdots a_k)^{-1} = a_k^{-1} \cdots a_1^{-1}$.\\

Then
$$
(a_1 \cdots a_{k+1})^{-1} = a_{k+1}^{-1}(a_1 \cdots a_k)^{-1} = a_{k+1}^{-1}a_k^{-1} \cdots a_1^{-1}
$$\\

Thus $P(n)$ is true for all $n \in \mathbb{N}$.

Now, if $n\geq0$ , let $a_1=a_2=...=a_n=a$\\

Then, we get $(a^n)^{-1}=(a^{-1})^n$
\end{proof}
\begin{problem}(1.26)
    A group in which $x^2=e$ for every $x$ must be abelian.
\end{problem}
\begin{proof}
    $x,y \in G \implies x^2y^2=e\implies xy=x^{-1}y^{-1}$\\
    \AK{Strategy:Using the idea of $ab^{-1}=e\implies a=b$ along with the given information which is not there for no reason.$x^2=e\implies x=x^{-1}$ so we can definitely try leveraging this property.}
    Now,
    $(xy)(yx)^{-1}=(xy)(x^{-1}y^{-1})=(xy)(xy)=e \implies xy=yx$
\end{proof}


\begin{problem} (1.27)\\

(i) Let $G$ be a finite abelian group containing no elements $a \ne e$ with $a^2 = e$. Evaluate 
$$
a_1 a_2 \cdots a_n,
$$ 
where $a_1, a_2, \dots, a_n$ is a list, with no repetitions, of all the elements of $G$.\\

(ii) Prove Wilson's Theorem: If $p$ is prime, then
$$
(p - 1)! \equiv -1 \pmod{p}.
$$ 
\end{problem}
\begin{proof}
(i)
    Claim: $a_1...a_n=e$\\

    \textbf{For $n$ being odd}, $\forall a_i \in G, \exists a_i ^{-1} \in G ,i \in \{1,..,n\}$. As the group is abelian and finite so only one element say $a_k \in G, k \in \{1,..,n\}$ remains. Now it is clear that $a_k=a_k^{-1}\implies a_k =e$ is the only possibility.\\
    Thus, $a_1...a_n=e$
    
    \AK{Strategy:For P3 (i)~~Let's start with small cases??? For $Z_n$ when $n$ is odd, the evaluation yields $0$, when $n$ is even, the evaluation yields $1,2,3,...$ The even case also seems to have an element whose order is $2$ which violates the condition. The odd case doesn't.So the naive conjecture seems that the evaluation would yield the identity element.}
    
    \textbf{For $n$ being even},
    as each element is distinct and each of their inverse is unique, so we get, for $e \in G, \exists a_k \in G, k\in \{1,..,n\}$ such that $e a_k =e \implies a_k =e$ but $e$ is unique so there is no group with order even satisfying the given conditions.\\
    
    Or we can use \textbf{Cauchy's theorem}, Let $G$ be a finite group and $p$ be a prime. If $p$ divides the order of $G$, then $G$ has a non identity element of order $p$.\\
    If $|G|$ was even then Cauchy's theorem implies that there is a non identity element of order $2$ which contradicts the hypothesis.\\

(ii)~~ We have $U(p)=\{1,2,...,p-1\}$. $U(p)$ is a finite abelian group. Now, each element of $U(p)$ has an inverse. $|U(p)|$ is even so there is a non identity element of order $2$.\\

For some $x \in U(p),x^2= 1 
 \implies x=x^{-1}\implies x=1,-1(=p-1)\implies x^{-1}=-1(=p-1),1$.

 As, we already know for a group with order even, eventually after pairing and cancellation,  $1 \cdot y=1\implies y=-1=p-1$.


$1.2....(p-1)=1.(p-1)$
 Thus,
 $$(p-1)! = p-1 = -1 $$
 

 \AK{We ignore the repeated use of $\equiv (\mod p)$ as it is clear that we are working in $\mod p$ environment due to the way $U(p)$ is defined.} 

\end{proof}

\begin{problem}
  Show that $\alpha : \mathbb{Z}_{11}\to \mathbb{Z}_{11}$, defined by $\alpha(x)=4x^2-3x^7$, is a permutation of $\mathbb{Z}_{11}$, 
  and write is as a product of disjoint cycles. What is the parity of $\alpha$? What about $\alpha^{-1}$?
\end{problem}
\begin{proof}
  
\end{proof}
\begin{Competition}
Is there a finite abelian group G such that the product of all the orders of its elements is $2^{2009}$?
\end{Competition}
\begin{proof}
  
\end{proof}




\subsection{Homomorphisms}



\section{\textbf{The Isomorphism Theorems}}
\subsection{Subgroups}

\subsection{Lagrange's Theorem}

\subsection{Cyclic Subgroups}

\subsection{Normal Subgroups}

\subsection{Quotient Subgroups}

\subsection{The Isomorphism Theorems}

\subsection{Correspondence Theorem}

\subsection{Direct Product}









\begin{bibdiv}
\begin{biblist}
\bib{rotman1995groups}{book}{
  title={An Introduction to the Theory of Groups},
  author={Rotman, Joseph J.},
  edition={4},
  series={Graduate Texts in Mathematics},
  publisher={Springer New York, NY},
  year={1995},
  isbn={978-0-387-94285-8, 978-1-4612-4176-8},
  doi={10.1007/978-1-4612-4176-8},
  note={Originally published by Allyn \& Bacon, 1965, 1973 and 1984}
}

\bib{moon2014homework}{misc}{
  author={Moon, Han-Bom},
  title={Homework},
  year={2014},
  howpublished={\url{https://www.hanbommoon.net/wp-content/uploads/2014/01/}},
  note={MATH 3005}
}



\end{biblist}
\end{bibdiv}
\vfill

\end{document}